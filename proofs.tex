\begin{frame}
    \frametitle{\mygreen{New Technique:} Updating Proof $\pi_i=c(q_i)$ After Change $(i, \delta_i)$}
    \pause
    We know $\pi_i' = \cm{q_i'}$, where:\pause
    \begin{align}
    \onslide<3->{q_i'(X) &=\frac{\phi'(X)-(v_i+\delta_i)}{X-i}}\\
    \onslide<4->{    & =\frac{\left(\phi(X) + \delta_i\lagr_i(X)\right) - v_i -\delta_i}{X-i}}\\
    \onslide<5->{    &=\frac{\phi(X) - v_i}{X-i}-\frac{\delta_i(\lagr_i(X)-1)}{X-i}}\\
    \onslide<6->{q_i'(X) &= q_i(X) + \delta_i\left(\frac{\lagr_i(X)-1}{X-i}\right)}
    \end{align}
    \pause[7]
    Applying KZG homomorphism, it follows that:\pause
    \begin{align}
    \pi_i'=\cm{q_i'}=\cm{q_i} \cdot \cm{\frac{\lagr_i(X)-1}{X-i}}^{\delta_i}
    \end{align}
    \pause
    \textbf{Thus,} each $\upk_i$ must include $\cm{\frac{\lagr_i(X)-1}{X-i}}$.\pause\xspace
    \textit{Can derive these from $g^{\tau^i}$'s!}
\end{frame}

\begin{frame}
    \frametitle{\mygreen{New Technique:} Updating Proof $\pi_i=c(q_i)$ After Change $(j, \delta_j)$, $j\ne i$}
    \small
    \pause
    Once again, we know $\pi_i' = \cm{q_i'}$, where:\pause
    \begin{align}
    \onslide<3->{q_i'(X) &=\frac{\phi'(X)-v_i}{X-i}}\\
    \onslide<4->{    &=\frac{\left(\phi(X) + \delta_j\lagr_j(X)\right) - v_i}{X-i}}\\
    \onslide<5->{    &=\frac{\phi(X) - v_i}{X-i}-\frac{\delta_j\lagr_j(X)}{X-i}}\\
    \onslide<6->{q_i'(X) &= q_i(X) + \delta_j\left(\frac{\lagr_j(X)}{X-i}\right)}
    \end{align}
    \pause[7]
    Applying KZG homomorphism, it follows that:\pause
    \begin{align}
    \pi_i'=\cm{q_i'}=\cm{q_i} \cdot \cm{\frac{\lagr_j(X)}{X-i}}^{\delta_j}
    \end{align}
    \pause
    \myred{\textbf{Problem:}} To update any $\pi_i$ after a change to $j$, need $\cm{\frac{\lagr_j(X)}{X-i}},\forall i\ne j$\pause\xspace $\Rightarrow O(\rn)$-sized $\upk_j$.\pause\\
    \mygreen{\textbf{Solution:}} Compute $\cm{\frac{\lagr_j(X)}{X-i}}$ in $O(1)$ time from information in $\upk_i$ and $\upk_j$.
\end{frame}

\begin{frame}
    \frametitle{\mygreen{New Technique:} Compute $c\left(\frac{\lagr_j(X)}{X-i}\right)$ in $O(1)$ Time}

    %\footnotesize
    \pause
    Let $A(X)=\prod_{i\in[0,n)} (X-i)$.\pause\xspace
    Then: % $\lagr_j(X) = \frac{A(X)}{A'(j)(X-j)}$,\pause\xspace
    %  (see \cref{s:partial-fraction-decomposition})
    \begin{align}
    %\onslide<3->{\frac{\lagr_j(X)}{X-i} &= \frac{A(X)}{A'(j)(X-j)(X-i)}}\\
    \label{eq:decomp1}
    \frac{\lagr_j(X)}{X-i} &= \frac{1}{A'(j)}\cdot \frac{A(X)}{(X-j)(X-i)}
    %\onslide<7->{    &= \frac{1}{A'(j)}\cdot W_{i,j}(X)}
    \end{align}
    Next, use \textbf{partial fraction decomposition} to rewrite:\pause
    % (see \cref{s:decompose-proof-update})
    \begin{align}
        \label{eq:decomp2}
        \frac{A(X)}{(X-i)(X-j)} &= \frac{1}{i-j}\cdot \frac{A(X)}{X-i} + \frac{1}{j-i}\cdot \frac{A(X)}{X-j}
    \end{align}
    \pause
    Now, replacing \cref{eq:decomp2} into \cref{eq:decomp1}:
    \begin{align}
        \frac{\lagr_j(X)}{X-i} &= \frac{1}{A'(j)}\cdot \left(\frac{1}{i-j}\cdot \frac{A(X)}{X-i} + \frac{1}{j-i}\cdot \frac{A(X)}{X-j}\right)
    \end{align}\pause
    As a result:\pause
    \begin{align}
        %\cm{\frac{A(X)}{(X-i)(X-j)}} &= \cm{\frac{A(X)}{X-i}}^{\frac{1}{i-j}} \cm{\frac{A(X)}{X-j}}^{\frac{1}{j-i}}\\
        \cm{\frac{\lagr_j(X)}{X-i}} &= \left(\cm{\frac{A(X)}{X-i}}^{\frac{1}{i-j}}\cdot \cm{\frac{A(X)}{X-j}}^{\frac{1}{j-i}}\right)^{\frac{1}{A'(j)}}
    \end{align}
    \pause
    \textbf{Thus,} each $\upk_i$ must include $\cm{\frac{A(X)}{X-i}}$ and $A'(i)$.\pause\xspace
    \textit{Can derive from $g^{\tau^i}$'s!}
\end{frame}
