%
% Styling
%
\newcommand{\api}{\hangindent=\parindent \hangafter=1 \noindent}
%\def\bitcoin{%
%    \leavevmode
%    \vtop{\offinterlineskip %\bfseries
%        \setbox0=\hbox{B}%
%        \setbox2=\hbox to\wd0{\hfil\hskip-.03em
%            \vrule height .3ex width .15ex\hskip .08em
%            \vrule height .3ex width .15ex\hfil}
%        \vbox{\copy2\box0}\box2}}
%
%\DeclareRobustCommand{\bitcoin}{{%
%        \normalfont\sffamily
%        \raisebox{-.25ex}{\makebox[.1\width][l]{-\kern-.3em-}}B%
%}}

%
% Some nice colors:
%  Blue: #268bd2
%  (Grey) Blue: #657b83
%  Green: #859900
%  (Dark) Orange: #cb4b16
%  Red: #dc322f
%  Periwinkle: #6c71c4
%  Pink: #d33682
%  Teal: #2aa198
%  Yellow: #b58900
%
\definecolor{myBlueColor}{HTML}{268BD2}
\definecolor{myYellowColor}{HTML}{B58900}
%\definecolor{myGreenColor}{HTML}{859900}
\definecolor{myRedColor}{HTML}{DC322F}
%\definecolor{myOrangeColor}{HTML}{CB4B16}
\newcommand{\myblue}[1]{\textcolor{myBlueColor}{#1}}
\newcommand{\myred}[1]{\textcolor{myRedColor}{#1}}
\newcommand{\myyellow}[1]{\textcolor{myYellowColor}{#1}}
%\newcommand{\mygreen}[1]{\textcolor{myGreenColor}{#1}}
\newcommand{\mygreen}[1]{\textcolor{mLightGreen}{#1}}
\newcommand{\myorange}[1]{\textcolor{myOrangeColor}{#1}}

%
% Notes and TODOs
%
\newcommand{\todo}[1]{\noindent\textcolor{red}{\textbf{(}\textsc{\textbf{{TODO: }}}}#1\textcolor{red}{\textbf{)}}}
\newcommand{\anote}[1]{\textcolor{blue}{[\textbf{Alin}: #1]}}

%
% Some mathy things
%
\DeclarePairedDelimiter{\ceil}{\lceil}{\rceil}
\DeclarePairedDelimiter{\floor}{\lfloor}{\rfloor}
\newcommand{\dom}{{Domain}}
\newcommand{\bezout}{B\'ezout\xspace}
\newcommand{\Gr}{\mathbb{G}}
\newcommand{\Gho}{\mathbb{G}_{?}}
\newcommand{\F}{\mathbb{F}}
\newcommand{\Fp}{\mathbb{F}_p}
\newcommand{\GT}{\mathbb{G}_T}
\newcommand{\Zp}{\mathbb{Z}_p}
\newcommand{\poly}{{poly}}
\newcommand{\negl}{{negl}}
\newcommand{\lagr}{\ensuremath{\mathcal{L}}}
\newcommand{\Adv}{\ensuremath{\mathcal{A}}\xspace} % adversary

%
% Vector Commitment things
%
\newcommand{\vect}[1]{\ensuremath{\vec{{#1}}}}

\newcommand{\vcsetup}{\ensuremath{{VC}}.\ensuremath{{KeyGen}}\xspace}
\newcommand{\vccommit}{\ensuremath{{VC}}.\ensuremath{{Commit}}\xspace}
\newcommand{\vccommitempty}{\ensuremath{{VC}}.\ensuremath{{EmptyCommit}}\xspace}
\newcommand{\vcopenpos}{\ensuremath{{VC}}.\ensuremath{{ProvePos}}\xspace}
\newcommand{\vcverifypos}{\ensuremath{{VC}}.\ensuremath{{VerifyPos}}\xspace}
\newcommand{\vcverifyaggpos}{\ensuremath{{VC}}.\ensuremath{{VerifyAggPos}}\xspace}
\newcommand{\vcverifyupk}{\ensuremath{{VC}}.\ensuremath{{VerifyUPK}}\xspace}
\newcommand{\vccommupdate}{\ensuremath{{VC}}.\ensuremath{{UpdDig}}\xspace}
\newcommand{\vcproofupdate}{\ensuremath{{VC}}.\ensuremath{{UpdProof}}\xspace}

\newcommand{\vcopenall}{\ensuremath{{VC}}.\ensuremath{{ProveAll}}\xspace}
\newcommand{\vcaggregateproofs}{\ensuremath{{VC}}.\ensuremath{{AggregateProofs}}\xspace}

\newcommand{\emptydigest}{d_\varnothing}
\newcommand{\cm}[1]{\ensuremath{\color{mLightGreen}c\left(\normalcolor#1\color{mLightGreen}\right)\normalcolor}}
\newcommand{\pp}{\ensuremath{{PP}}\xspace}
\newcommand{\prk}{\ensuremath{\mygreen{prk}}\xspace}
\newcommand{\vrk}{\ensuremath{\mygreen{vrk}}\xspace}
\newcommand{\upk}{\ensuremath{\mygreen{upk}}\xspace}

\newcommand{\treepath}{\ensuremath{{treepath}}}

%
% \Sys things
%
\newcommand{\addr}{\ensuremath{{addr}}\xspace}
\newcommand{\bal}{\ensuremath{{bal}}\xspace}
\newcommand{\cnt}{\ensuremath{{cnt}}\xspace}
\newcommand{\inittxn}{\texttt{\textbf{INIT}}\xspace}
\newcommand{\initspendtxn}{\texttt{INITSPEND}\xspace}
\newcommand{\minttxn}{\texttt{\textbf{MINT}}\xspace}
\newcommand{\pk}{\ensuremath{\mathbf{{pk}}}\xspace}
\newcommand{\PK}{\ensuremath{{PK}}\xspace}
%\newcommand{\spendtxn}{\texttt{\textbf{SPEND}}\xspace}
\newcommand{\txfer}{\texttt{\textbf{TXFER}}\xspace}
\newcommand{\tpk}{\ensuremath{{tpk}}\xspace}
\newcommand{\tsk}{\ensuremath{{tsk}}\xspace}
\newcommand{\tx}{\ensuremath{{tx}}\xspace}

% Use this to animate a PDF with a bunch of pages!
% Adds the pages #1 through #2 (inclusive) from pdfFile=#3 as separate slides, each with includegraphicsArgs=#4.
% Sets the slide's title to #5 and adds #6 and #7 as before and after text.
\def\pdfslides#1#2#3#4#5#6#7{%
    \foreach \index in {#1, ..., #2}{%
        \begin{frame}%
            \frametitle{#5}%
            {#6}%
            %\pause
            \begin{center}%
                \includegraphics[page=\index,#4]{#3}%
            \end{center}%
            {#7}
        \end{frame}%
}}