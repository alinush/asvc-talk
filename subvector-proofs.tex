\begin{frame}
    \frametitle{Computing $I$-subvector Proofs From Scratch}

    \footnotesize
    \pause
    Can use \alert{KZG batch proofs} to compute a constant-sized \alert{$I$-subvector} proof $\pi_I$ for all $(v_i)_{i\in I}$:\pause
    \begin{align}
    \onslide<3->{A_I(X) &=\prod_{i\in I} (X - i)}\\
    \onslide<4->{R_I(X) &=\sum_{i\in I} v_i\cdot \lagr_i^*(X),\ \text{where}\  \lagr_i^*(X)=\prod_{j\in I,j\ne i}\frac{X - j}{i - j}}\\
    \onslide<5->{q_I(X) &= \frac{\phi(X)-R_I(X)}{A_I(X)}}
    \end{align}
    \pause[6]
    \begin{alertblock}{Note that}
    \pause
    $A_I(i)=0$ and $R_I(i) = v_i$, $\forall i\in I$.\pause\\
    Can compute $A_I(X)$ and $R_I(X)$ in $O(|I|\log^2{|I|})$ field operations.\pause\\
    Can compute $q_I(X)$ in $O(n\log{n})$ field operations.\pause
    \end{alertblock}
    $I$-subvector proof is $\pi_I = g^{q_I(\tau)}$, computed with $O(n-|I|)$-sized multiexp.\pause\xspace
    To verify:\pause
    \begin{align}
    e(c/g^{R_I(\tau)}, g) = e(\pi_I, g^{A_I(\tau)})
    \end{align}
    \pause
    \textit{Note:} \vrk contains $(g^{\tau^i})_{i\in[0,|I|]}$ %, which are needed for verifier to commit to $A_I(X)$ and $R_I(X)$.
\end{frame}

\begin{frame}
    \frametitle{Aggregating $I$-subvector Proof $\pi_I$ From $(\pi_i)_{i\in I}$}

    \footnotesize
    \pause
    We use \textit{partial fraction decomposition}, as proposed by Drake and Buterin~\cite{Buterin20UsingPoly} (see \cref{s:partial-fraction-decomposition}):\pause
    \begin{align}
    \frac{1}{A_I(X)} = \frac{1}{\prod_{i\in I} (X-i)} = \sum_{i\in I} \frac{1}{A_I'(i)} \cdot \frac{1}{X-i}
    \end{align}
    \pause
    We ``decompose'' the quotient $q_I(X)=\frac{\phi(X)-R_I(X)}{A_I(X)}$ in $\pi_I$ as:\pause % a linear combination of the $q_i(X)$'s in the $\pi_i$'s:\pause
    \begin{align}
    \onslide<5->{q_I(X) &= \phi(X)\frac{1}{A_I(X)}- R_I(X)\frac{1}{A_I(X)}}\\
    \onslide<6->{    &= \phi(X)\frac{1}{A_I(X)}- \left(\sum_{i\in I} v_i\cdot \lagr_i^*(X)\right)\frac{1}{A_I(X)}}\\
    \onslide<7->{    &= \phi(X)\sum_{i\in I} \frac{1}{A_I’(i)(X-i)} - \left(\sum_{i\in I} v_i\cdot \frac{A_I(X)}{A_I’(i)(X-i)}\right)\cdot \frac{1}{A_I(X)}} \\
    \onslide<8->{    &= \sum_{i\in I} \frac{\phi(X)}{A_I’(i)(X-i)} - \sum_{i\in I} \frac{v_i}{A_I’(i)(X-i)}}\\
    \onslide<9->{    &= \sum_{i\in I} \frac{1}{A_I’(i)}\cdot \frac{\phi(X) - v_i}{X-i}}\\
    \onslide<10->{   &= \sum_{i\in I} \frac{1}{A_I’(i)}\cdot q_i(X)}
    \end{align}
\end{frame}


\begin{frame}
    \frametitle{Aggregating $I$-subvector Proof $\pi_I$ From $(\pi_i)_{i\in I}$ (Continued)}

    \pause
    To aggregate $\pi_I$:\pause

    \textbf{Step 1:} Interpolate $A_I(X)=\prod_{i\in I} (X-i)$ in $O(|I|\log^2{|I|})$ field operations.\pause\\
    \textbf{Step 2:} Compute its derivative $A_I'(X)$ in $O(|I|)$ field operations.\pause\\
    \textbf{Step 3:} Compute all $A_I'(i)$ in $O(|I|\log^2{|I|})$ field operations via a \textit{polynomial multipoint evaluation}~\cite{vG13ModernCh10}.\pause\\
    \textbf{Step 4:} Compute $\pi_I$ using an $O(|I|)$-sized multiexp:\pause
    \begin{align}
    \pi_I = \prod_{i\in I} \pi_i^{1/A_I'(i)}
    \end{align}
    \pause
    \textbf{Job done:} Can aggregate $\pi_i$'s for $(v_i)_{i\in I}$ into constant-sized $\pi_I$.
\end{frame}